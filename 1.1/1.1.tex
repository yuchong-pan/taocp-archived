
\documentclass[letterpaper, reqno,11pt]{article}
\usepackage[margin=1.0in]{geometry}
\usepackage{color,latexsym,amsmath,amssymb}
\usepackage{fancyhdr}
\usepackage{amsthm}
\usepackage{graphicx}

\newcommand{\RR}{\mathbb{R}}
\newcommand{\CC}{\mathbb{C}}
\newcommand{\ZZ}{\mathbb{Z}}
\newcommand{\QQ}{\mathbb{Q}}
\newcommand{\NN}{\mathbb{N}}
\newcommand{\st}{\ \mathrm{s.t.}\ }
\newcommand{\with}{\textrm{ with }}
\newcommand{\wehave}{\textrm{, we have }}
\pagestyle{fancy}
\lhead{TAOCP Section 1.1 Exercises}
\rhead{Yuchong Pan}
\begin{document}
\pagenumbering{arabic}
\title{TAOCP Section 1.1 Exercises}
\author{Yuchong Pan}
\date{\today}
\newtheorem{thm}{Theorem}
\newtheorem{lemma}{Lemma}
\newcommand{\algo}[2]{{\bf Algorithm #1.} #2}
\newcommand{\algostep}[3]{{\bf #1.} [#2] #3}
\newcommand{\algoend}[0]{$\blacksquare$}
\maketitle
%

\noindent {\bf 1.} $t\leftarrow a,\ a\leftarrow b,\ b\leftarrow c,\ c\leftarrow d,\ d\leftarrow t$.

\medskip

\noindent {\bf 2.}
\begin{proof}
    Consider 2 cases, $m>n$ and $m\leq n$ for original inputs. \\
    For the first case (i.e., $m>n$ originally), we have $0\leq r<n$ after step E1. \\
    If $r=0$, then the algorithm terminates in step E2. \\
    Otherwise, since $r<n$, we have $n<m$, or $m>n$, after step E3. \\
    For the second case (i.e., $m\leq n$ originally), we have $0\leq r\leq m$ after E1. \\
    If $r=0$, then $m=n$ and the algorithm terminates in step E2. \\
    Otherwise, $r=m<n$, so we have $n<m$, or $m>n$, after step E3. \\
    Therefore, step E3 always gives $m>n$, so $m$ is always $n$ at the beginning of step E1, except possibly the first time this step occurs.
\end{proof}

\medskip

\noindent {\bf 3.}
\algo{F}{Given two positive integers $m$ and $n$, find their greatest common divisor, i.e., the largest positive integer that evenly divides both $m$ and $n$.} \\
\algostep{F1}{Find remainder.}{Divide $m$ by $n$ and let $r$ be the remainder. Set $m\leftarrow r$. (We will have $0\leq m<n$.)} \\
\algostep{F2}{Does $m$ equal zero?}{If $m=0$, the algorithm terminates; $n$ is the answer.} \\
\algostep{F3}{Exchange.}{Exchange $m\leftrightarrow n$. (We will have $m>n>0$.)} \algoend \\

\medskip

\noindent {\bf 4.} Let $m=2166$, and let $n=6099$. \\
\begin{tabular}{c c c}
    $m$ & $n$ & $r$ \\
    \hline
    2166 & 6099 & 2166 \\
    6099 & 2166 & 1767 \\
    2166 & 1767 & 399 \\
    1767 & 399 & 171 \\
    399 & 171 & 57 \\
    171 & 57 & 0
\end{tabular} \\
Therefore, 57 is the greatest common divisor of 2166 and 6099.

\medskip

\noindent {\bf 5.} The "Procedure for Reading This Set of Books" fails to meet the following three features of algorithms.
\begin{itemize}
    \item Finiteness: as shown in the flow chart, the procedure never terminates because readers will return to Chapter 1 after they finish all of the 12 chapters.
    \item Output: the procedure does not have an output that is returned to readers.
    \item Effectiveness: some steps of the procedure are not effective; e.g., "work exercises" is not effective to most readers because they may be stuck in some exercise problems (forever).
\end{itemize}

\medskip

\noindent {\bf 6.} As stated in the text, only the remainder of $m$ after division by $n$ is relevant. \\
Therefore, we will find $T_5$ by trying the algorithm for $m=1,2,3,4,5$.
\begin{itemize}
    \item $m=1$. \\
        \begin{tabular}{c | c c c}
            \# & $m$ & $n$ & $r$ \\
            \hline
            1 & 1 & 5 & 1 \\
            2 & 5 & 1 & 0
        \end{tabular}
    \item $m=2$. \\
        \begin{tabular}{c | c c c}
            \# & $m$ & $n$ & $r$ \\
            \hline
            1 & 2 & 5 & 2 \\
            2 & 5 & 2 & 1 \\
            3 & 2 & 1 & 0
        \end{tabular}
    \item $m=3$. \\
        \begin{tabular}{c | c c c}
            \# & $m$ & $n$ & $r$ \\
            \hline
            1 & 3 & 5 & 3 \\
            2 & 5 & 3 & 2 \\
            3 & 3 & 2 & 1 \\
            4 & 2 & 1 & 0
        \end{tabular}
    \item $m=4$. \\
        \begin{tabular}{c | c c c}
            \# & $m$ & $n$ & $r$ \\
            \hline
            1 & 4 & 5 & 4 \\
            2 & 5 & 4 & 1 \\
            3 & 4 & 1 & 0
        \end{tabular}
    \item $m=5$. \\
        \begin{tabular}{c | c c c}
            \# & $m$ & $n$ & $r$ \\
            \hline
            1 & 5 & 5 & 0
        \end{tabular}
\end{itemize}
Therefore, $T_5=\frac{2+3+4+3+1}{5}=\frac{13}{5}$.

\medskip

\noindent {\bf 7.} Consider $m<n$. \\
Since $m$ is fixed, the contributions of the cases for which $m\geq n$ are negligible to $U_m$. \\
Let $r$ be the remainder of $m$ after division by $n$, and we have $r=m$. \\
After the first execution of Algorithm E, we set $m'\leftarrow n,\ n'\leftarrow r=m$. \\
Let $r'$ be the remainder of $m'$ after division by $n'$. \\
Since $m'=n$ and $n'=m$, then $r'$ is the remainder of $n$ after division by $m$. \\
After the second execution of Algorithm E, we set $m''\leftarrow n'=m,\ n''\leftarrow r'$. \\
Therefore, after the first two executions of Algorithm E, only the remainder of $n$ after division by $m$ is relevant, so we can find $U_m$ by trying the algorithm for $n=1,2,\ldots,m$. \\
Hence, $U_m$ is well defined. \\
Recall that after the first execution of Algorithm E, we set $m'\leftarrow n,\ n'\leftarrow m$. \\
Since $m$ is known and $n$ is allowed to range over all positive integers, then $n'$ is known and $m'$ is allowed to range over all positive integers. \\
Therefore, we have $U_m=T_{n'}+1=T_m+1$.

\medskip

\noindent {\bf 8.} Let $N=2$. \\
Define $\theta_j,\phi_j,a_j,b_j$ for $0\leq j<N$ as follows. \\
\begin{tabular}{c | c c c c}
    $j$ & $\theta_j$ & $\phi_j$ & $a_j$ & $b_j$ \\
    \hline
    0 & $a^mb^{m+1}$ & $a^mb$ & 1 & 0 \\
    1 & $a^{n+1}b^n$ & $ab^n$ & 2 & 0 \\
\end{tabular}

\medskip

\noindent {\bf 9.} Let $C_1=\left(Q_1,I_1,\Omega_1,f_1\right)$ and $C_2=\left(Q_2,I_2,\Omega_2,f_2\right)$ be computational methods. \\
Suppose that there exists a surjective function $F\colon\ Q_2\rightarrow Q_1$ such that for all $q_1,q_2\in Q_2$ with $f_2\left(q_1\right)=q_2$, we have either $F\left(q_1\right)=F\left(q_2\right)$ or $f_1\left(F\left(q_1\right)\right)=F\left(q_2\right)$. \\
Suppose furthermore that for all $i\in I_2$, $F(i)\in I_1$, and for all $\omega\in\Omega_2$, $F(\omega)\in\Omega_1$. \\
Then we say that "$C_2$ is a representation of $C_1$" or "$C_2$ simulates $C_1$".

\end{document}
